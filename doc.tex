\documentclass[12pt]{article}
\usepackage{amsmath}
\usepackage{amsfonts}
\usepackage{graphicx}
\usepackage{hyperref}

\title{Methodology for Infection Spread Model and Optimization}
\author{}
\date{}

\begin{document}

\maketitle

\section{Overview}
The model described in the provided code simulates the spread of an infection across a grid, optimized to match a reference infection map using an agent-based model (ABM). The infection dynamics are governed by parameters that affect transmission rates, recovery rates, and the influence of environmental factors. The model is implemented using PyTorch for efficient computation, and the parameters are optimised through a gradient-based approach.

\section{Model Components}
\subsection{Grid Initialization}
The grid represents a two-dimensional space with each cell representing an individual. Each individual can be in one of three compartments:
\begin{itemize}
    \item \textbf{Susceptible (S)}: Individuals not yet infected.
    \item \textbf{Infected (I)}: Individuals currently infected.
    \item \textbf{Recovered (R)}: Individuals who have recovered from the infection.
\end{itemize}
At the start of the simulation, an initial infection map $I_0$ is provided, which defines the initial distribution of infected individuals ($I=I_0$). The grid is represented as a tensor $\mathbf{grid}$ with three channels, one for each compartment. The susceptible individuals are initialized to $S = 1 - I_0$, and the recovered individuals are initialized to $R = 0$.

\subsection{Force of Infection Calculation}
The force of infection $\zeta$ for each cell is computed based on the infection status of its neighbours and the infection transmission weights. The transmission weights are computed using a function that depends on the distance to the neighbours and environmental factors such as plant density. We define a (sparse weight) matrix for the $i^{\textrm{th}}$ cell $W_i$ through:
\[
W_{ij} = \beta \exp\left\{ - \left (\frac{r_{ij}}{\sigma} \right)^\alpha \right\}
\]
where:
\begin{itemize}
    \item $r_{ij}$ is the distance between cell $i$ and its neighbor $j$.
    \item $\beta$ and $\sigma$ are parameters controlling the strength and decay of the transmission.
\end{itemize}

We can all an anisotropic models where the infection spread is influenced by direction, an additional factor is introduced to weight the effective distance such that:
\[
W_{ij}=W(r_{ij}^{\star},\alpha, \beta, \sigma), \quad r_{ij}^{\star}=r_{ij}/(1+V |\cos \theta|)
\]
where:
\begin{itemize}
    \item $V$ acts as a imposed 'velocity', promoting the spread in directions parallel (and anti-parallel) to $\theta$, where $\theta$ is the angle between the direction of infection spread and the vector from cell $i$ to neighbour $j$.
\end{itemize}

The infection force $\zeta_i$ at each cell $i$ is calculated as the weighted sum of the infection status of over all cells in the domain, excluding the self interaction:
\[
\zeta_{i, i\ne j} = \sum_j W_{ij} I_j
\]
where $I_j$ is the infection status of neighbour $j$.

\subsection{Infection and Recovery Dynamics}
The infection dynamics are modelled using the following processes:
\begin{itemize}
    \item \textbf{Infection (S $\to$ I)}: Susceptible individuals can become infected with probability $P_{\text{inf}}$:
    \[
    P_{\text{inf}} = 1 - \exp(-\zeta)
    \]
    where $\zeta$ is the force of infection.
    \item \textbf{Recovery (I $\to$ R)}: Infected individuals can recover with probability $P_{\text{rec}}$:
    \[
    P_{\text{rec}} = \gamma
    \]
    where $\gamma$ is the recovery rate.
\end{itemize}
\subsection{Gumbel-Softmax Approximation for State Transitions}

To model the transitions in a differentiable way, we use the Gumbel-Softmax trick. Given a probability \( P \) for a transition (e.g., \( P_{\text{inf}} \) for infection or \( P_{\text{rec}} \) for recovery), the standard approach would be to sample from a Bernoulli distribution:
\[
Z \sim \text{Bernoulli}(P).
\]
However, this sampling is not differentiable, which poses issues for gradient-based optimization. Instead, we employ the Gumbel-Softmax approximation, which enables reparameterizable sampling.

\subsubsection{Gumbel Distribution}
The Gumbel distribution is used to transform uniform random variables into samples that approximate categorical sampling. If \( U \sim \text{Uniform}(0,1) \), then the Gumbel-distributed random variable is defined as:
\[
G = -\log(-\log U).
\]
This allows us to introduce randomness in a way that can be backpropagated.

\subsubsection{Gumbel-Softmax Approximation}
To approximate a Bernoulli sample, we define two logits: one for the event happening (\( P \)) and one for the event not happening (\( 1 - P \)). We introduce independent Gumbel noise \( G_1 \) and \( G_2 \) and compute:

\[
\tilde{P} = \frac{\exp\left((\log P + G_1)/\tau\right)}{\exp\left((\log P + G_1)/\tau\right) + \exp\left((\log (1-P) + G_2)/\tau\right)}
\]

where \( \tau \) is a temperature parameter that controls the degree of approximation. As \( \tau \to 0 \), \( \tilde{P} \) approaches a hard threshold at 0 or 1.

\subsubsection{Interpreting the Transition}
For state transitions, such as \( S \to I \) (infection), we replace the discrete Bernoulli sample with:

\[
\xi = \tilde{P}_{\text{inf}} S.
\]

Similarly, for recovery:

\[
\eta = \tilde{P}_{\text{rec}} I.
\]

These terms are used in the continuous update equations:

\[
\Delta S = -\xi S, \quad \Delta I = \xi S - \eta I, \quad \Delta R = \eta I.
\]

This formulation ensures differentiability, enabling backpropagation and optimization via gradient descent.

\subsection{Optimization Process}
The optimization process aims to find the best parameters that minimize the discrepancy between the simulated infection map and a reference infection map. The parameters include the transmission rates ($\alpha$, $\beta$), recovery rate ($\gamma$), and advective velocity ($V$). The optimization is performed using the Adam optimizer and a loss function based on structural similarity index measure (SSIM) or mean squared error (MSE).

The loss function is computed as:
\[
\mathcal{L}= \frac{1}{N} \sum_{i=1}^{N} (I_{\text{sim}} - I_{\text{ref}})^2
\]
where $I_{\text{sim}}$ and $I_{\text{ref}}$ are the simulated and reference infection maps. Adam updates the parameters as follows:
\begin{align*}
m_t &= \beta_1 m_{t-1} + (1 - \beta_1) \nabla_\Theta \mathcal{L} \\
v_t &= \beta_2 v_{t-1} + (1 - \beta_2) (\nabla_\Theta \mathcal{L})^2 \\
\hat{m}_t &= \frac{m_t}{1 - \beta_1^t}, \quad \hat{v}t = \frac{v_t}{1 - \beta_2^t} \\
\Theta_t &= \Theta_{t-1} - \frac{\eta}{\sqrt{\hat{v}_t} + \epsilon} \hat{m}_t
\end{align*}
where $m_t$ and $v_t$ are biased first and second moment estimates, $\eta$ is the learning rate, and $\epsilon$ is a small constant for numerical stability.
\subsection{Numerical Simulations and Results}
The model is run for a set number of timesteps $n_{\text{timesteps}}$. For each timestep, the force of infection and the transition probabilities are updated, and the grid is evolved according to the infection and recovery dynamics. The model is then evaluated against the reference infection map using the loss function.

\end{document}
